\documentclass[12pt,a4paper,oneside]{article}             % Single-side
%\documentclass[11pt,a4paper,twoside,openright]{report}  % Duplex

%\PassOptionsToPackage{chapternumber=Huordinal}{magyar.ldf}
\usepackage{t1enc}
\usepackage[latin2]{inputenc}
\usepackage[magyar]{babel}
\usepackage[T1]{fontenc}
\usepackage[table]{xcolor}
\usepackage{makecell}
\usepackage{array}
\usepackage{mathtools}
\usepackage{amsmath}
\usepackage{amssymb}
\usepackage{enumerate}
\usepackage[thmmarks]{ntheorem}
\usepackage{graphics}
\usepackage{epsfig}
\usepackage{listings}
\usepackage{color}
%\usepackage{fancyhdr}
\usepackage{lastpage}
\usepackage{anysize}
\usepackage{sectsty}
\usepackage{setspace}  % Ettol a tablazatok, abrak, labjegyzetek maradnak 1-es sorkozzel!
\usepackage[hang]{caption}
\usepackage{hyperref}
\usepackage{indentfirst}
\usepackage[normalem]{ulem}
\usepackage{ctable}
\usepackage{cleveref}
\usepackage{mathabx}
\usepackage{verbatim}
\usepackage{tocloft}
\usepackage{float}
\usepackage{rotating}
\usepackage{pdflscape}
\usepackage{lipsum}
\usepackage{everypage}

%\usepackage[paper=portrait,pagesize]{typearea}

%--------------------------------------------------------------------------------------
% Code sytles
%--------------------------------------------------------------------------------------

% Python -----------------------------------------------------
\definecolor{weborange}{RGB}{255,165,0}
\lstdefinestyle{Python}{
    language        = Python,
    basicstyle      = \fontsize{11}{11}\ttfamily,
		%backgroundcolor=\color{white!10}
    keywordstyle    = \color{weborange},
    emphstyle				= \color{blue},
    keywordstyle    = [2] \color{teal}, % just to check that it works
    stringstyle     = \color{green},
    commentstyle    = \color{red}\ttfamily
}
\lstnewenvironment{pycode}{
\lstset{style=Python}
\lstset{literate={'}{{'}}1}}{}


% Python -----------------------------------------------------
\definecolor{mygreen}{RGB}{28,172,0} % color values Red, Green, Blue
\definecolor{mylilas}{RGB}{170,55,241}

\lstdefinestyle{Matlab}{
		language=Matlab,%
    basicstyle=\fontsize{10}{11}\ttfamily,
    %basicstyle=\color{red},
    breaklines=true,%
    morekeywords={matlab2tikz},
    keywordstyle=\color{blue},%
    morekeywords=[2]{1}, keywordstyle=[2]{\color{black}},
    identifierstyle=\color{black},%
    stringstyle=\color{mylilas},
    commentstyle=\color{mygreen},%
    showstringspaces=false,%without this there will be a symbol in the places where there is a space
    numbers=left,%
    numberstyle={\tiny \color{black}},% size of the numbers
    numbersep=9pt, % this defines how far the numbers are from the text
    emph=[1]{for,end,break},emphstyle=[1]\color{red}, %some words to emphasise
    %emph=[2]{word1,word2}, emphstyle=[2]{style},    
}

%--------------------------------------------------------------------------------------
% Main variables
%--------------------------------------------------------------------------------------
\newcommand{\vikszerzoone}{Horny�k M�t� D�niel}
\newcommand{\vikszerzotwo}{Moln�r Marcell}
\newcommand{\vikszerzothr}{Poleczki �kos}
\newcommand{\vikcim}{H�zi feladat}
\newcommand{\vikcimm}{Felhaszn�l�i dokument�ci�}
\newcommand{\viktanszek}{M�r�stechnika �s Inform�ci�s Rendszerek Tansz�k}
\newcommand{\vikdoktipus}{Be�gyazott rendszerek szoftvertechnol�gi�ja}
%\newcommand{\vikdepartmentr}{Moln�r Marcell}

%--------------------------------------------------------------------------------------
% Page layout setup
%--------------------------------------------------------------------------------------
% we need to redefine the pagestyle plain
% another possibility is to use the body of this command without \fancypagestyle
% and use \pagestyle{fancy} but in that case the special pages
% (like the ToC, the References, and the Chapter pages)remain in plane style

\pagestyle{plain}
%\setlength{\parindent}{0pt} % �ttekinthet�bb, angol nyelv� dokumentumokban jellemz�
%\setlength{\parskip}{8pt plus 3pt minus 3pt} % �ttekinthet�bb, angol nyelv� dokumentumokban jellemz�
\setlength{\parindent}{12pt} % magyar nyelv� dokumentumokban jellemz�
\setlength{\parskip}{0pt}    % magyar nyelv� dokumentumokban jellemz�

\chapterfont{\large\upshape\bfseries}
\marginsize{35mm}{25mm}{15mm}{15mm} % anysize package
\setcounter{secnumdepth}{0}
\sectionfont{\large\upshape\bfseries}
\setcounter{secnumdepth}{2}
\singlespacing
\frenchspacing

%--------------------------------------------------------------------------------------
%	Setup hyperref package
%--------------------------------------------------------------------------------------
\hypersetup{
    bookmarks=true,            % show bookmarks bar?
    unicode=true,             % non-Latin characters in Acrobat�s bookmarks
    pdftitle={\vikcim},        % title
    pdfauthor={\vikszerzoone, \vikszerzotwo, \vikszerzothr},    % author
    pdfsubject={\vikdoktipus}, % subject of the document
    pdfcreator={\vikszerzoone, \vikszerzotwo, \vikszerzothr},   % creator of the document
    pdfproducer={Producer},    % producer of the document
    pdfkeywords={keywords},    % list of keywords
    pdfnewwindow=true,         % links in new window
    colorlinks=true,           % false: boxed links; true: colored links
    linkcolor=black,           % color of internal links
    citecolor=black,           % color of links to bibliography
    filecolor=black,           % color of file links
    urlcolor=black             % color of external links
}

%--------------------------------------------------------------------------------------
% Set up listings
%--------------------------------------------------------------------------------------
\lstset{
	basicstyle=\scriptsize\ttfamily, % print whole listing small
	keywordstyle=\color{black}\bfseries\underbar, % underlined bold black keywords
	identifierstyle=, 					% nothing happens
	commentstyle=\color{white}, % white comments
	stringstyle=\scriptsize\sffamily, 			% typewriter type for strings
	showstringspaces=false,     % no special string spaces
	aboveskip=3pt,
	belowskip=3pt,
	columns=fixed,
	backgroundcolor=\color{lightgray},
} 		
\def\lstlistingname{lista}	

%--------------------------------------------------------------------------------------
%	Some new commands and declarations
%--------------------------------------------------------------------------------------
\newcommand{\code}[1]{{\upshape\ttfamily\scriptsize\indent #1}}

% define references
\newcommand{\figref}[1]{\ref{fig:#1}.}
\newcommand{\equatref}[1]{(\ref{eq:#1})}
\newcommand{\listref}[1]{\ref{listing:#1}.}
\newcommand{\sectref}[1]{\ref{sect:#1}}
\newcommand{\tabref}[1]{\ref{tab:#1}.}

\DeclareMathOperator*{\argmax}{arg\,max}
%\DeclareMathOperator*[1]{\floor}{arg\,max}
\DeclareMathOperator{\sign}{sgn}
\DeclareMathOperator{\rot}{rot}
\definecolor{lightgray}{rgb}{0.95,0.95,0.95}

\author{\vikszerzo}
\title{\viktitle}
\includeonly{
	fedlap,%
	chapter1,%
}
%--------------------------------------------------------------------------------------
%	Setup captions
%--------------------------------------------------------------------------------------
\captionsetup[figure]{
%labelsep=none,
font={normalsize,it},
%justification=justified,
width=.75\textwidth,
aboveskip=10pt}

\renewcommand{\captionlabelfont}{\small\bf}
\renewcommand{\captionfont}{\footnotesize\it}

%--------------------------------------------------------------------------------------
% Page numbering for landscape pages
%--------------------------------------------------------------------------------------

\newcommand{\Lpagenumber}{\ifdim\textwidth=\linewidth\else\bgroup
  \dimendef\margin=0 %use \margin instead of \dimen0
  \ifodd\value{page}\margin=\oddsidemargin
  \else\margin=\evensidemargin
  \fi
  \raisebox{\dimexpr -\topmargin-\headheight-\headsep-0.5\linewidth}[0pt][0pt]{%
    \rlap{\hspace{\dimexpr \margin+\textheight+\footskip}%
    \llap{\rotatebox{90}{\thepage}}}}%
\egroup\fi}
\AddEverypageHook{\Lpagenumber}%
%--------------------------------------------------------------------------------------


%--------------------------------------------------------------------------------------
% Table of contents and the main text
%--------------------------------------------------------------------------------------
\begin{document}
\singlespacing

\pagenumbering{arabic}
\onehalfspacing

%--------------------------------------------------------------------------------------
%	The title page
%--------------------------------------------------------------------------------------
\begin{titlepage}
\begin{center}
\includegraphics[width=60mm,keepaspectratio]{figures/BMElogo.png}\\
\vspace{0.3cm}
\textbf{Budapesti M�szaki �s Gazdas�gtudom�nyi Egyetem}\\
\textmd{Villamosm�rn�ki �s Informatikai Kar}\\
\textmd{\viktanszek}\\[5cm]

\vspace{0.4cm}
{\huge \bfseries \vikcim}\\[0.8cm]
\vspace{0.5cm}
\textsc{\Large \vikdoktipus}\\[4cm]

\emph{K�sz�tett�k:}\\
\textmd{\vikszerzoone}\\
\textmd{\vikszerzotwo}\\
\textmd{\vikszerzothr}\\

\vfill
{\large \today}
\end{center}
\end{titlepage}


\vfill
%----------------------------------------------------------------------------
\section{A feladat r�vid �sszefoglal�sa}
%----------------------------------------------------------------------------
A megval�s�tand� feladat egy k�tdimenzi�s, oldaln�zetes l�v�ld�z�s j�t�k l�trehoz�sa Android oper�ci�s rendszer alatt, mely a k�vetkez�k�ppen m�k�dik.

Egyszerre k�t j�t�kos j�tszik egym�s ellen, �sszesen fejenk�nt h�rom �letponttal, websocket alap� kapcsolaton kereszt�l. A j�t�k c�lja az ellenf�l �letpontjainak lenull�z�sa. A j�t�k elind�t�sakor a felhaszn�l� egy felugr� men�pontban adhatja meg a csatlakozni k�v�nt szerver IP c�m�t, illetve lehet�s�ge van ezen k�v�l az alapvet� be�ll�t�sokat v�ltoztatni a j�t�k megkezd�se el�tt. Teh�t a vez�rl�si m�dok �s az egy�ni �zl�snek megfelel� h�tt�r kiv�laszt�s�ra, valamint a h�tt�rzene ki-be kapcsol�s�ra.  A be�ll�t�sok elv�gz�s�vel �s a csatlakoz�s gomb megnyom�s�val a felhaszn�l� csatlakozik a megadott szerverhez. A be�ll�t�sokat az alkalmaz�s k�t ind�t�sa k�z�tt is meg�rzi. Amennyiben mindk�t j�t�kos sikeresen csatlakozott, a j�t�k egy visszasz�ml�l�st k�vet�en elkezd�dik.

A be�ll�t�sok alkalm�val megadott vez�rl� m�dok alapj�n vagy gombokkal vagy joystick seg�ts�g�vel van lehet�s�g a karakter mozgat�s�ra, az ellenf�l megsebz�s�re pedig egy dedik�lt nyom�gomb szolg�l. A nyom�gomb lenyom�sakor egy sz�ml�l� indul el, mely n�h�ny m�sodpercig blokkolja az �jrat�zel�st, ezzel megakad�lyozva sorozatos t�mad�st. A j�t�k dinamik�j�nak kisz�les�t�s�nek �rdek�ben a blokkol�son k�v�l akad�lyok is elhelyez�sre ker�lnek a p�ly�n, amelyekre ak�r fel is lehet ugorni a karakterrel.

%----------------------------------------------------------------------------
\section{A program szerkezeti egys�gei}
%----------------------------------------------------------------------------
Az egyes class-ok t�bbf�lek�ppen csoportos�that�k. Funkci�k tekintet�ben n�gy nagy csoportot k�l�nb�ztethet�nk meg:
\begin{itemize}
  \item \textbf{a j�t�k grafikus elemeit kezel� oszt�lyok,}
  \item \textbf{a h�l�zat kezel�s�re szolg�l� oszt�lyok,}
  \item \textbf{a program �llapotg�p�t kezel� oszt�lyok,}
  \item \textbf{a GUI kezel�s��rt felel�s oszt�lyok.}
\end{itemize}

Az Android k�rnyezet �ltal megk�l�nb�ztetett csoportok:
\begin{itemize}
  \item \textbf{hagyom�nyos oszt�lyok,}
  \item \textbf{activity oszt�lyok.}
\end{itemize}

Funkcionalit�s alapj�n teh�t n�gy alapvet� oszt�lyt�pust k�l�nb�ztet�nk meg. A tov�bbiakban az egyes csoportok tagjai ker�lnek bemutat�sra funkcionalit�s �s m�s oszt�lyokkal val� kapcsolat alapj�n.


%----------------------------------------------------------------------------
\section{Grafikus elemeket vez�rl� oszt�lyok}
%----------------------------------------------------------------------------
\textbf{GameObject oszt�ly}
\begin{itemize}
  \setlength\itemsep{0mm}
  \item Az oszt�ly feladata a p�ly�n elhelyezked� alapvet� objektumok kezel�se (karakterek, akad�lyok, l�ved�kek, stb.).
  \item Minden j�t�k objektum ebb�l az �soszt�lyb�l sz�rmazik le.
  \item Funkci�i pl.: x, y poz�ci� megad�sa, sz�less�g �s magass�g megad�sa az adott objektumokra vonatkoz�an.
\end{itemize}

\textbf{Character oszt�ly}
\begin{itemize}
  \setlength\itemsep{0mm}
  \item A j�t�kosok avat�rjait megtestes�t� oszt�ly, mely a GameObject-b�l sz�rmazik le.
  \item Alapvet� feladata a karakterek dinamik�j�nak kezel�se, a poz�ci� friss�t�se, az �let kezel�se, illetve a karakterek kirajzol�sa.
\end{itemize}

\textbf{Opponent oszt�ly}
\begin{itemize}
  \setlength\itemsep{0mm}
  \item Funkci�j�t tekintve a Character oszt�llyal megegyez� oszt�ly.
  \item A Character oszt�lyb�l sz�rmazik.
  \item A poz�ci� friss�t�se a websocket fel�l �rkez� adatok alapj�n t�rt�nik.
\end{itemize}

\textbf{Bullet oszt�ly}
\begin{itemize}
  \setlength\itemsep{0mm}
  \item Feladata a t�zel�skor kil�tt goly�k kezel�se pl.: tal�lat vizsg�lat, kirajzol�s.
  \item A GameObject-b�l sz�rmazik.
\end{itemize}

\textbf{Platform oszt�ly}
\begin{itemize}
  \setlength\itemsep{0mm}
  \item Feladata a p�ly�n elhelyezett akad�lyok kirajzol�sa �s annak vizsg�lata, hogy az egyes karakterek rajta �llnak-e vagy sem.
  \item Szint�n a GameObject-b�l sz�rmazik.
\end{itemize}

%----------------------------------------------------------------------------
\section{Kommunik�ci��rt felel�s oszt�lyok}
%----------------------------------------------------------------------------

\textbf{Connector oszt�ly}
\begin{itemize}
  \setlength\itemsep{0mm}
  \item Feladata a kapcsolat ki�p�t�se k�t Android eszk�z k�z�tt, illetve az IP c�m bevitel�re szolg�l� GUI kirajzol�s�t is elv�gzi.
  \item Kommunik�l a MainActivity �s a SettingsActivity oszt�lyokkal.
\end{itemize}

\textbf{Websocket oszt�ly}
\begin{itemize}
  \setlength\itemsep{0mm}
  \item A h�l�zaton val� adatk�ld�sre �s adatfogad�sra szolg�l� oszt�ly.
\end{itemize}

%----------------------------------------------------------------------------
\section{GUI kezel� oszt�lyok}
%----------------------------------------------------------------------------

\textbf{MainActivity oszt�ly}
\begin{itemize}
  \setlength\itemsep{0mm}
  \item Az AppCompatActivity Android �soszt�lyb�l sz�rmazik le.
  \item A j�t�k vez�rl�s�hez sz�ks�ges nyom�gombokat, joystick-ot, �s a hozz�juk tartoz� esem�nyek kezel�s��rt felel�s.
  \item GUI kezel�sen k�v�l ebben az oszt�lyban ker�l implement�l�sra a h�tt�rzene kezel�se.
\end{itemize}

\textbf{SettingsActivity oszt�ly}
\begin{itemize}
  \setlength\itemsep{0mm}
  \item A j�t�k indul�sakor megadhat� be�ll�t�sok kezel�s��rt felel�s.
  \item GUI funkci�kat is ell�t pl.: be�ll�t�sok bevitel��rt felel�s gombok, switchek kezel�se.
  \item A MainActivity oszt�ly sz�m�ra szolg�ltatja a konfigur�ci�s inform�ci�kat.
\end{itemize}
	
%----------------------------------------------------------------------------
\section{Egy�b seg�doszt�lyok}
%----------------------------------------------------------------------------

\textbf{CountDown oszt�ly}
\begin{itemize}
  \setlength\itemsep{0mm}
  \item Az IP c�m megad�sakor, ha az ellenf�l j�t�kos is r�csatlakozott a szerverre egy sz�ml�l� indul el, mely a j�t�k kezdet�t jelenti.
  \item A be�p�tett AsyncTask oszt�lyb�l sz�rmazik.
  \item Blokkolja a j�t�kost a mozg�s �s l�v�s lehet�s�g�t�l egy ,,Dialog" ablakkal, am�g el nem indul a j�t�k.
\end{itemize}

\textbf{Constants oszt�ly}
\begin{itemize}
  \setlength\itemsep{0mm}
  \item A be�ll�t�sok nyilv�ntart�s�ra szolg�l� absztrakt oszt�ly.
\end{itemize}
\pagebreak

%----------------------------------------------------------------------------
\section{A rendszert le�r� UML diagram}
%----------------------------------------------------------------------------

\begin{figure}[!h]
\hspace*{-2.5cm}
\includegraphics[width=190mm,keepaspectratio]{figures/uml.png}
\caption{A programhoz tartoz� UML diagram}
\label{fig:control}
\end{figure}


\label{page:last}
\end{document}
